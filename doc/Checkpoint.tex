\documentclass[11pt]{article}

\usepackage{fullpage}
\usepackage{etoolbox}
\usepackage{graphicx}
\graphicspath{ {./assets/emulate_diagram.png/} }


\AtBeginEnvironment{quote}{\itshape}

\begin{document}
\title{ARM11 Checkpoint}
\author{Mithun Sri, Shweta Banerjee, Devin Fernando, Taeho Kim}

\maketitle

\section{Group Organisation}


\-\hspace{1cm}On the 30th of May, our group had an initial offline meeting to discuss how to split up the work. We decided that it would be best for each individual to work on one or two instructions so Mithun was assigned the {\bf Single Data Transfer Instruction}; {\bf Branch / Multiply Instruction} was given to Devin; Shweta and Taeho, responsible for the {\bf Data Processing Instruction}. This was to ensure everyone had fair amount of workload. \\
\-\hspace{1cm}We then discussed what data types and structures to use for the registers, the memory, the instructions, etc. We also agreed on a same coding style convention (GTK+) from early on so we don't spend time refactoring the code later. \\
\-\hspace{1cm}After completing our assigned work, we met again offline to put our implementation together. We factored out the common helper function across the files, worked together on the functionalities to load, and execute the instructions. \\
\-\hspace{1cm}Our group is having regular meetings via Teams, constantly checking our individual progress to ensure that no one is falling behind and everyone is on track. We also created a group chat as mean of communication, where we frequently ask questions if we are unsure of how to implement certain functionalities or if we don't understand parts of the specification. So far, we have helped and learned a lot from each other.\\ 
\-\hspace{1cm}Although, as a team, we are working quite effectively and efficiently, we could definitely make some improvements for the remaining part of the C Group Project. Below are our thoughts on how our team can improve in the future.

\begin{quote}
  "We should start our work a bit early next time." -Taeho
\end{quote}
\begin{quote}
  "I think it would be nice if we can..." -Devin
\end{quote}
\begin{quote}
  "I think we work more efficiently during our in person group meetings, so maybe it is better to have more in person meetings and fewer online meetings." -Shweta
\end{quote}
\begin{quote}
  "We should get more used to git version control." -Mithun
\end{quote}


\section{Implementation Strategies}

We used \texttt{uint8\_t}, \texttt{uint16\_t}, and \texttt{uint32\_t} instead of int as this guarantees the same number of bits are used on every platform. As the values stored will always be positive, we used an unsigned representation rather than the signed representation. This is also something we would want to keep doing to improve memory efficiency of our program. 

\begin{center}
    \includegraphics[scale=0.45]{assets/emulate_diagram.png}
\end{center}

Below is a brief explanation of each file – note an instruction is executed if and only if the condition succeeds. 
\begin{itemize}
    \item \texttt{choose\_instruction.c} - This file is responsible for processing the instruction and then calling the right source file to execute the instruction. 
    \item \texttt{multiply.c} - If the accumulate bit is set, it stores the value of $\texttt{rm} \times \texttt{rs}$ + \texttt{rn} in the register \texttt{rd}, where \texttt{rm}, \texttt{rs}, and \texttt{rn} represent the values stored in registers \texttt{rm}, \texttt{rs}, and \texttt{rn} respectively. However, if accumulate is not set, then it stores $\texttt{rm} \times \texttt{rs}$ in \texttt{rd}. It updates CPSR flags if the S bit is set. 
    \item \texttt{single\_data\_transfer.c} - The file contains code which performs either a load instruction or a store instruction depending on the L bit. Then an offset value is calculated which can either be an immediate or another register - this offset value is used to shift the register. A pre-indexing offset or post-indexing offset is also performed which is determined by the P bit. 
    \item \texttt{data\_processing.c} - Depending on the \texttt{opcode}, one of ten different operations are performed using the contents of register rn and \texttt{Operand2}. The value of \texttt{Operand2} is determined by the state of the I flag – \texttt{Operand2} is either a rotated immediate constant or a shifted register. The CPSR flags are updated if the S bit of the instruction is set. 
    \item \texttt{branch.c} - The offset is extracted, shifted to the left by two bits and sign extended. Then the next instruction is fetched from the \texttt{pc} and the offset is added to it. This next instruction then gets put into the \texttt{pc} again. 
    \item \texttt{emulate\_utilities.c} - This file contains helper functions, each of which are needed by more than one file. This prevents code duplication and improves efficiency and this is something we will want to do for the assembler as well. 
    
\end{itemize}

\end{document}
